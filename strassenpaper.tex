\documentclass[10pt]{article}
\usepackage[margin = 1in]{geometry}
\usepackage{amsmath}
\usepackage{amsthm}
\usepackage{enumerate}
\usepackage{graphicx}
\usepackage{caption}

\linespread{1.2}

\parskip 8pt           % sets spacing between paragraphs
%\renewcommand{\baselinestretch}{1.5} % Uncomment for 1.5 spacing between lines
\parindent 0pt		 % sets leading space for paragraphs
\title{ \vspace{-4ex} Optimizing Matrix Multiplication\vspace{-1ex}}
\author{Carl Gao \& Michelle Deng\\ CS 124 -- Programming Assignment 2}
\date{ \vspace{-3ex} \today}
\newtheorem{lemma}{Lemma}
\newtheorem*{obs}{Observation}
\newtheorem*{ass}{Assumption}
\newtheorem*{cor}{Corollary}
\newcommand{\lp}{\left(}
\newcommand{\rp}{\right)}

\newcommand{\tl}{\tilde} 
\begin{document}
\maketitle

 \vspace{-.4in}

\section{Introduction}

Suppose we want to multiply two $n$-by-$n$ matrices $A = \{a_{ij}\}$ and $B = \{b_{ij}\}$, where the subscript $ij$ denotes the element at the $i$th row and $j$th column of a matrix. By definition, the product of $A$ and $B$ is the $n$-by-$n$ matrix $C = \{c_{ij}\}$, where
\begin{equation}
c_{ij} = \sum_{k=1}^n a_{ik}b_{kj}.
\end{equation}

The conventional matrix multiplication algorithm simply uses the above formula to compute each $c_{ij}$ individually. Each $c_{ij}$ requires $n$ scalar multiplications and $n-1$ additions, and there are $n^2$ elements in $C$, giving a total of $n^2(n+n-1) = 2n^3-n^2$ arithmetic operations. Assuming that such primitive arithmetic operations take constant time, which is realistic when $a_{ij}$ and $b_{ij}$ are not extremely large, and that all other operations (e.g. data-copying, memory access, etc.) are free, the overall computation time $T_c(n)$ using the conventional algorithm on an $n$-by-$n$ matrix is 
\begin{equation}
T_c(n) = 2n^3 - n^2 = O(n^3).
\end{equation} 

Famously, Strassen provides an $O(n^{\log_27}) \approx O(n^{2.81})$ recursive algorithm for matrix multiplication, which is asymptotically more efficient for large $n$. The algorithm breaks the $n$-by-$n$  multiplication into seven matrix multiplications and 10 matrix additions using various $\frac{n}{2}$-by-$\frac{n}{2}$ quadrants of $A$ and $B$, generating seven $\frac{n}{2}$-by-$\frac{n}{2}$ auxiliary matrices $P_1, ..., P_7$. These $P$ matrices are then recombined in a sequence of 8 $\frac{n}{2}$-by-$\frac{n}{2}$ matrix additions to generate the four quadrants of $C$.\footnote{Lecture Notes 8 contains a complete description of the algorithm and definitions of these variables/matrices.} Since an $n$-by-$n$ matrix addition requires $n^2$ constant-time additions, the total computation time $T_s(n)$ using Strassen's algorithm can be described by the following recurrence:
\[T_s(n) = 7T\lp \frac{n}{2} \rp + 18\lp \frac{n}{2} \rp ^2 \]
which is $O (n^{\log_27})$ by the master theorem.

Note that we have not specified a base case for the recursion. Strassen's algorithm is only \emph{asymptotically} more efficient than the conventional algorithm; for small $n$, the conventional algorithm is faster. Thus, to optimize matrix multiplication, one may want to use Strassen's algorithm whenever $n \ge n_0$, and the conventional algorithm for $n < n_0$. That is, $n_0$ is the cross-over point between the two methods, and the full recurrence for Strassen's is
\begin{equation}
T_s(n) = 
\begin{cases}
7T\lp \frac{n}{2} \rp + 18\lp \frac{n}{2} \rp ^2 &\text{for $n \ge n_0$}\\
T_c(n) & \text{for $n < n_0$}
\end{cases}
\end{equation}
We will first estimate $n_0$ analytically and then experimentally. 
\\

\section{Analytical estimate of $n_0$}

To estimate $n_0$, we will use the simplifications described above, where primitive arithmetic operations are constant time, and all other operations are free. However, in a real-world implementation, Strassen's recursive algorithm requires many more memory-access and data-copying steps than the conventional algorithm, so this bound is likely an underestimate of any empirical cross-over point. For simplicity, we will assume that $n$ is even.

The cross-over point $n_0$ is defined as the smallest integer such that $T_c(n_0) > T_s(n_0)$. In particular, 
\begin{align*}
T_c(n_0) &> T_s(n_0)\\
2n_0^3 - n_0^2 &> 7T\lp \frac{n_0}{2} \rp + 18\lp \frac{n_0}{2} \rp ^2\\
&= 7\lp2\lp \frac{n_0}{2} \rp^3 - \lp \frac{n_0}{2} \rp^2\rp + 18\lp \frac{n_0}{2} \rp ^2\\
8n_0^3 - 4n_0^2 &> 7n_0^3 + 11 n_0^2 \\
0 &> n_0^3 - 15 n_0^2 = n_0^2(n_0-15)
\end{align*}
Hence $n_0 > 15$, and our analytic estimate is $n_0 = 16$.

\section {Implementation}

\subsection{Reusing temporary arrays}
First of all, we worked with a single 

\subsection{Dealing with odd $n$}
We used a dynamic padding approach, in which a single extra layer of 0's was added to odd matrices at any point in the recursion and even matrices were left alone. This enabled much higher space efficiency than static padding (to a power of 2) while keeping program complexity low. 

\subsection{Cache efficiency}




\section {Conclusion}





\end{document}
